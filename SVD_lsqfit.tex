\documentclass[a4paper]{article}
% General document formatting
\usepackage[margin=0.7in]{geometry}
\usepackage[parfill]{parskip}
\usepackage[utf8]{inputenc}

\usepackage{amsmath,amssymb,amsfonts,amsthm}
\usepackage{empheq,etoolbox}
\usepackage{xcolor}


% Update display of subequation numbering (Xy) > (X.y)
\patchcmd{\subequations}% <cmd>
{\theparentequation\alph{equation}}% <search>
{\theparentequation.\alph{equation}}% <replace>
{}{}% <success><failure>



\begin{document}
The matrix $X \in \mathbb{R}^{n\times m}$ contains the surface elevation of $n$ grid cells (rows) for $m$ different training geometries (columns) representing the modelling output at several time steps and from different ensemble runs.
\begin{equation}
    X =
    \begin{bmatrix}
        x_{1\,(M_1)} & x_{1\,(M_2)} & ... & x_{1\,(M_m)}\\
        x_{2\,(M_1)} & x_{2\,(M_2)} & ... & x_{2\,(M_m)}\\
        ...          & ...          & ... & ...     \\
        x_{n\,(M_1)} & x_{n\,(M_2)} & ... & x_{n\,(M_m)}
    \end{bmatrix}
\end{equation}
where $M_i$ is the $i^{th}$ dataset generated by the model.
We want to find the vector $\mathbf{x}_\mathrm{rec}$ of reconstructed surface elevation data at a certain point in time, of which we only have incomplete knowledge with most of the values unknown
\begin{equation}
    \mathbf{x}_\mathrm{rec} =
    \begin{bmatrix}
        x_1\\
        x_2\\
        ...\\
        x_n       
    \end{bmatrix}_{\mathrm{rec}}
\end{equation}
Take the SVD of $X$, truncated to $r$ singular values, where $r<m$:
\begin{equation}
    X \approx U \Sigma V^T
\end{equation}
The columns of $U \in \mathbb{R}^{n\times r}$ are the left singular vectors and in this case they represent the 'eigenglaciers' of each of the $r$ modes. The matrix $\Sigma \in \mathbb{R}^{r\times r}$ is a diagonal matrix of the corresponding singular values, and the columns of $V \in \mathbb{R}^{m\times r}$ are the right singular vectors that, for each of the $m$ modelling datasets, assign coefficients to the $r$ eigenglacier modes. Thus, if $\mathbf{x}_\mathrm{rec}$ is in the 'hull' of $X$, we can find coefficients comprising a vector $\mathbf{v}_\mathrm{rec}$ such that
\begin{equation}
    \mathbf{x}_\mathrm{rec} = U \Sigma \mathbf{v}_\mathrm{rec}
\end{equation}
However, since we do not know all elements of $\mathbf{x}_\mathrm{rec}$ we cannot solve for $\mathbf{v}_\mathrm{rec}$ directly. Instead, we can set up a new problem including only the points in $\mathbf{x}$ where we actually have data:
\begin{equation}
    \mathbf{x}_\mathrm{data} = A\mathbf{v}_\mathrm{rec}
\end{equation}
where $\mathbf{x}_\mathrm{data} \in \mathbb{R}^d$ is a vector of the the available data (with $d<n$) and the rows of $A\in \mathbb{R}^{d\times r}$ are the corresponding rows of the matrix $U \Sigma$.

The number of known data points is in the order of $d \sim 10^5$ and therefore much larger than $r$, the number of coefficients we want to solve for.
It is thus an overdetermined system for which we can find the regularized least squares solution
\begin{equation} \label{eq:min_prob}
    \min_{\mathbf{v_\mathrm{rec}}}    ||\mathbf{x}_\mathrm{data}- A \mathbf{v}_\mathrm{rec}||_2^2 + \lambda ||\mathbf{v}_\mathrm{rec}||_2^2
\end{equation}
where $\lambda$ is the regularization parameter. Taking the gradient of Eq.~\eqref{eq:min_prob} and setting it to zero results in the normal equation
\begin{equation} \label{eq:normal_eq}
    (A^TA+\lambda I)\hat{\mathbf{v}}_\mathrm{rec} = A^T \mathbf{x}_\mathrm{data}~.
\end{equation}
% Inserting back $A=U\Sigma$:
% \begin{align}
%     (\Sigma^T U^T U \Sigma + \lambda I)\hat{\mathbf{v}}_\mathrm{rec} &= \Sigma^T U^T \mathbf{x}_\mathrm{rec}\\
%     \label{eq:unitary}
%     (\Sigma^T \Sigma + \lambda I)\hat{\mathbf{v}}_\mathrm{rec} &= \Sigma^T U^T \mathbf{x}_\mathrm{rec}
% \end{align}
% e do not actually know all the elements of 


% The matrix $(\Sigma^T \Sigma + \lambda I)$ is diagonal and thus invertible, so we can directly solve the system:
% \begin{equation}
%     \hat{\mathbf{v}}_\mathrm{rec} = (\Sigma^T \Sigma + \lambda I)^{-1} \Sigma^T U^T \mathbf{x}_\mathrm{rec}~.
% \end{equation}
% However, 

In order to solve the system analytically, we take a full SVD of $A$
\begin{equation}
    A = U_A \Sigma_A V_A^T
\end{equation}
and insert it into \eqref{eq:normal_eq} to obtain
\begin{align}
    (V_A\Sigma_A^T U_A^T U_A \Sigma_A V_A^T + \lambda I)\hat{\mathbf{v}}_\mathrm{rec} &= V_A\Sigma_A^T U_A^T \mathbf{x}_\mathrm{data}\\
    \label{eq:unitary}
    (V_A\Sigma_A^T \Sigma_A V_A^T + V_A \lambda I V_A^T)\hat{\mathbf{v}}_\mathrm{rec} &= V_A\Sigma_A^T U_A^T \mathbf{x}_\mathrm{data}\\
    V_A(\Sigma_A^T \Sigma_A + \lambda I) V_A^T \hat{\mathbf{v}}_\mathrm{rec} &= V_A\Sigma_A^T U_A^T \mathbf{x}_\mathrm{data}\\
\end{align}
where we used the unitary properties of $U_A$ and $V_A$, i.e. $U_A^T U_A = U_A U_A^T= I$ and $V_A^T V_A = V_A V_A^T= I$.
Introducing $\mathbf{z} = V_A^T \hat{\mathbf{v}}_\mathrm{rec}$ we get
\begin{equation}
    (\Sigma_A^T \Sigma_A + \lambda I) \mathbf{z} = \Sigma_A^T U_A^T \mathbf{x}_\mathrm{data}
\end{equation}
where $(\Sigma_A^T \Sigma_A + \lambda I)$ is a diagonal matrix. The solution of the system is thus
\begin{equation}
    \mathbf{z} = (\Sigma_A^T \Sigma_A + \lambda I)^{-1} \Sigma_A^T U_A^T \mathbf{x}_\mathrm{data}~.
\end{equation}
Finally, we can substitute this solution into $\hat{\mathbf{v}}_\mathrm{rec} = V_A \mathbf{z}$ to obtain a least squares solution for $\hat{\mathbf{v}}_\mathrm{rec}$:
\begin{equation}
    \hat{\mathbf{v}}_\mathrm{rec} = V_A (\Sigma^T \Sigma + \lambda I)^{-1} \Sigma^T U^T \mathbf{x}_\mathrm{data}~.
\end{equation}


% For Eq.~\eqref{eq:unitary} to hold, $U$ must be unitary, i.e. $U^TU = U U^T = I$, which is true for a truncated SVD but not when parts of the rows of $U$ are removed corresponding to grid cells where there is no data in $\mathbf{x}_\mathrm{rec}$.



\end{document}